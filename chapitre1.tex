\chapter{Preliminary study}
\lhead{\textit{Chapitre \thechapter}}
\rhead{\textit{Premier chapitre}}

\section{Introduction}
Lorem ipsum dolor sit amet, consectetur adipisicing elit, sed do eiusmod
tempor incididunt ut labore et dolore magna aliqua. Ut enim ad minim veniam


\section{General Definitions}
\subsection{Website}
A website is a collection of publicly accessible, interlinked Web pages that share 
a single domain name. Websites can be created and maintained by an individual, group, 
business or organization to serve a variety of purposes.\cite{Techopedia2011-zo}

\subsection{Responsive Web Design}
Responsive web design (RWD) is a web development approach that creates dynamic 
changes to the appearance of a website, depending on the screen size and orientation of 
the device being used to view it. RWD is one approach to the problem of designing for 
the multitude of devices available to customers, ranging from tiny phones to huge desktop monitors.\cite{noauthor_undated-an}

\subsection{Interactive Course}
The term "Interactive Course" (IC) typically describes material of an educational nature 
delivered in a format which allows the user to directly impact the materials content, pace, and out-come.
An example of such material would be a computer based presentation requiring a user to select the correct 
answer to a give question before proceeding to the next topic.
These types of courses are almost always computer based and most likely to be delivered to the user thru the internet. 
Due to their convenient delivery, availability and almost endless subject matter, 
interactive courses have become a major tool for those seeking to provide as well as those seeking to obtain education, 
training or certification in a given area of study.\\
With growing access and availability to computers and the internet, many schools, universities, 
businesses and government agencies are turning to IC's to train and educate their students and staff.\cite{noauthor_undated-dt}

\subsection{Computer Architecture}
\subsubsection{In general}
Computer architecture is a science or a set of rules stating how computer software and hardware are 
joined together and interact to make a computer work. It not only determines how the computer works 
but also of which technologies the computer is capable.\cite{Pam2018-so}
\subsubsection{In our case}
Computer architecture or better known as "Structure Machine"(STRM) is a course for first-year students in the Math and Computer Science department (MI) 
of the university bouira AMO. and also in all the MI departments in Algeria universities.\\ \\
\indent Computer architecture (STRM) has a coefficient of 3 and a credit of 5, which makes it one of the more fundamental courses.





%%Bla bla bla bla bla (voir Figure~\ref{fig:Allmagne}).

%+++++++++++++++++++++++++++++++++++++++++++++++++++++++++++++ 
 \begin{figure} [h!]%[htbp]
 	\vspace*{13pt}
 	\subfloat{\label{}\includegraphics[scale=0.43]{img/Allmagne.png}} 
 	\vspace*{13pt}               
 	\caption{Titre figure 1} 
 	\label{fig:Allmagne}
 \end{figure} 
%+++++++++++++++++++++++++++++++++++++++++++++++++++++++++++++++++++++++++++++++++++


%h, here
%t, top
%b, bottom


\subsubsection{Une sous sous section}
Lorem ipsum dolor sit amet, consectetur adipisicing elit, sed do eiusmod
tempor incididunt ut labore et dolore magna aliqua. Ut enim ad minim veniam,
quis nostrud exercitation ullamco laboris nisi ut aliquip ex ea commodo
consequat. Duis aute irure dolor in reprehenderit in voluptate velit esse
cillum dolore eu fugiat nulla pariatur. Excepteur sint occaecat cupidatat non
proident, sunt in culpa qui officia deserunt mollit anim id est laborum.

\paragraph{Un paragraphe}
Lorem ipsum dolor sit amet, consectetur adipisicing elit, sed do eiusmod
tempor incididunt ut labore et dolore magna aliqua. Ut enim ad minim veniam,
quis nostrud exercitation ullamco laboris nisi ut aliquip ex ea commodo
consequat. Duis aute irure dolor in reprehenderit in voluptate velit esse
cillum dolore eu fugiat nulla pariatur. Excepteur sint occaecat cupidatat non
proident, sunt in culpa qui officia deserunt mollit anim id est laborum.

\section{Une deuxième section principale}
Lorem ipsum dolor sit amet, consectetur adipisicing elit, sed do eiusmod
tempor incididunt ut labore et dolore magna aliqua. Ut enim ad minim veniam,
quis nostrud exercitation ullamco laboris nisi ut aliquip ex ea commodo
consequat. Duis aute irure dolor in reprehenderit in voluptate velit esse
cillum dolore eu fugiat nulla pariatur. Excepteur sint occaecat cupidatat non
proident, sunt in culpa qui officia deserunt mollit anim id est laborum.

\subsection{Une sous section}
Lorem ipsum dolor sit amet, consectetur adipisicing elit, sed do eiusmod
tempor incididunt ut labore et dolore magna aliqua. Ut enim ad minim veniam,
quis nostrud exercitation ullamco laboris nisi ut aliquip ex ea commodo
consequat. Duis aute irure dolor in reprehenderit in voluptate velit esse
cillum dolore eu fugiat nulla pariatur. Excepteur sint occaecat cupidatat non
proident, sunt in culpa qui officia deserunt mollit anim id est laborum.

\subsection{Exemple d'un tableau}

See Table \ref{tab:tableau1}

\begin{table}[h!]
\begin{center}
	\begin{tabular}{|l|l|}
		\hline
		\textbf{Température en C} & \textbf{Température en F} \\
		\hline
		\hline
		0 & ... \\
		\hline
		1 & ... \\
		\hline
		3 & ... \\
		\hline
		... & ... \\
		\hline
	\end{tabular}
\end{center}
\caption{Titre du tableau}
\label{tab:tableau1}
\end{table}

